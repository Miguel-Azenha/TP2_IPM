Aós a realização dos testes de usabilidade e a nálise das observações dos utilizadores sobre o protótipo de baixa fidelidade, foi desenvolvido um protótipo de alta fidelidade no Figma, incorporando as melhorias identificadas. Este proróripo representa uma versão mais polida, interativa e visualmente consistente da aplicação original, matendo o foco na simplicidade e clareza das interações.

O protótipo foi desenvolvido com o Figma, utilizando recursos de componentes reutilizáveis, auto layout, estilos de texto e cor, e interações animadas entre ecrãs.

Começámos por recriar as interfaces do protótipo de baixa fidelidade, iniciando pela \textit{bottom navigation bar}, elemento que garante uma navegação simples, previsível e coerente entre as diferentes secções da aplicação.

\begin{figure}[H]
    \centering
    \includegraphics[width=0.4\textwidth]{assets/Imagens/bottom_navbar.png}
    \caption{Opções da \textit{bottom navigation bar}}
\end{figure}

A \textit{bottom navigation bar} apresenta quatro opções principais, permitindo ao utilizador deslocar-se facilmente entre as funcionalidades centrais da aplicação:

\begin{itemize}
    \item Extrato – opção já existente na aplicação Navegante original, que apresenta o histórico de transações e carregamentos efetuados pelo utilizador;
    \item Cartões – funciona como o ecrã inicial da aplicação, exibindo todos os passes e bilhetes ativos e permitindo selecionar rapidamente o título a utilizar;
    \item Mais – agrega funcionalidades adicionais, como definições, ajuda e contacto com o suporte técnico;
    \item Perfil – direciona o utilizador para a interface de gestão de conta e dados pessoais.
\end{itemize}

De seguida realizámos a interface principal e as interfaces de compra de cartão ou bilhetes, mudando a cor dos botões de compra e leitura de cartão para diferenciar as diferentes opções e não induzir o utilizador a erro.

\begin{figure}[H]
    \centering
    \includegraphics[width=0.4\textwidth]{assets/Imagens/Hi_Fi_IleituraComprar.png}
    \caption{Opções da \textit{Interface de compra/leitura}}
\end{figure}

A interface de compra de bilhete tem a opção de selecionar o número de quantidade de bilhetes desejados e mostra a validade do bilhete. Ao carregar num passe na interface principal somo rederecionados para a interface com toda a informação do bilhete mostrando também as viagens que foram feitas e a que horas.

\begin{figure}[H]
    \centering
    \begin{subfigure}[t]{0.3\textwidth}
        \centering
        \includegraphics[width=\textwidth]{assets/Imagens/Hi_Fi_IcompraBilhete.png}
        \caption{Interface de compra de bilhete}
    \end{subfigure}
    \begin{subfigure}[t]{0.3\textwidth}
        \centering
        \includegraphics[width=\textwidth]{assets/Imagens/Hi_Fi_IinformacaoPasse.png}
        \caption{Interface de informações de passe}
    \end{subfigure}
    \caption{Interfaces de compra de bilhete e informação de passe}
\end{figure}

As restantes interfaces mantêm-se inalteradas em relação ao protótipo de baixa fidelidade, uma vez que não foram identificadas necessidades de modificação significativas.  
No entanto, apresenta-se abaixo o link de acesso ao protótipo de alta fidelidade desenvolvido no Figma, que permite navegar entre os diferentes ecrãs e interações da aplicação:

\begin{center}
\url{https://www.figma.com/design/rDVs9pUVSpS0LzFoTXjj0P/IPM?node-id=0-1&p=f&t=qCBJRD2fHeHu0ayR-0}
\end{center}

Nesta fase desenvolvemos o protótipo com a principal preocupação em mente de resolver os problemas de navegação, apesar de poucos, que foram encontrados diferenciando os botões por cores e tornando as interfaces mais percetivas. 