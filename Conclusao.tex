
A realização deste segundo trabalho prático permitiu dar continuidade ao estudo iniciado na primeira fase, passando da análise do problema e das necessidades dos utilizadores para a geração de soluções concretas e a sua materialização através da ideação e da prototipagem de baixa fidelidade.


A solução proposta procura eliminar a dependência do cartão físico, simplificando tarefas quotidianas como o carregamento e a validação do passe. Este conceito demonstra não só um potencial de melhoria da experiência do utilizador, como também um alinhamento com as tendências atuais de mobilidade digital e integração tecnológica nos transportes públicos.

O protótipo de baixa fidelidade revelou-se uma ferramenta essencial para comunicar ideias, testar fluxos de interação e identificar pontos de melhoria sem necessidade de recorrer a soluções técnicas complexas. O seu caráter flexível e iterativo contribuiu para um design mais centrado no utilizador e preparado para evoluir com base em futuros testes.

A transição do protótipo de baixa para alta fidelidade permitiu transformar um conceito funcional num produto visualmente apelativo e tecnicamente coerente.
As melhorias aplicadas reforçaram a intuitividade, eficiência e acessibilidade da aplicação, refletindo o contributo direto dos testes com utilizadores.

O protótipo de alta fidelidade constitui, assim, a base para um futuro desenvolvimento real da aplicação Navegante Digital, alinhado com as práticas de design centrado no utilizador.
