\subsection{Interface Principal }
  

\begin{figure}[H]
    \centering
    \includegraphics[width=0.3\textwidth]{assets/Imagens/Passes.png}
    \caption{Passes disponíveis}
\end{figure}

Ao abrir a aplicação, o utilizador é imediatamente apresentado a uma visão geral dos seus cartões e passes válidos, dispostos de forma clara e acessível.  
Este ecrã foi concebido para proporcionar uma perceção imediata do estado atual do transporte, eliminando a necessidade de múltiplas navegações ou menus secundários.

Cada passe ou cartão apresentado é interativo: ao selecionar um deles, o utilizador é redirecionado para uma nova interface dedicada, onde pode consultar todos os detalhes associados, tais como o tipo de passe, validade, saldo e histórico de utilização.  
Simultaneamente, esta ação ativa a funcionalidade de \textbf{validação digital}, permitindo que o telemóvel funcione como substituto do cartão físico para validação nas máquinas de transporte.

A interface inclui também um botão de acesso rápido a \textbf{bilhetes ou passes expirados}, oferecendo ao utilizador uma forma simples de identificar quais necessitam de recarregamento.  
Este componente fornece ainda alertas visuais que indicam o estado dos bilhetes únicos, informando se já foram utilizados ou se expiraram, promovendo assim uma gestão mais eficiente e intuitiva dos títulos de transporte.

Para utilizadores novos ou menos experientes, está disponível um botão de ajuda designado \textbf{“Como funciona?”}.  
Ao ser selecionado, este botão conduz a uma interface informativa que explica, de forma simples e ilustrada, as principais funcionalidades da aplicação e o modo de validação digital.  
Esta opção foi incluída como uma boa prática de usabilidade, reduzindo a curva de aprendizagem e promovendo a autonomia do utilizador.

Adicionalmente, a interface principal dispõe de um botão \textbf{“Comprar”}, que direciona o utilizador para a área de aquisição de novos bilhetes ou passes (descrita em detalhe noutra secção).  
Na parte superior do ecrã encontra-se o ícone de \textbf{Perfil}, que, ao ser selecionado, abre uma interface dedicada às informações pessoais e preferências do utilizador.

Por fim, na parte inferior da interface, foi implementada uma \textbf{barra de navegação inferior} (\textit{bottom navigation bar}) com três opções principais:
\begin{itemize}
    \item \textbf{Extrato} – apresenta o histórico de transações, carregamentos e utilizações recentes;
    \item \textbf{Cartões} – exibe todos os passes e bilhetes ativos, sendo o separador aberto por defeito;
    \item \textbf{Mais} – agrega funcionalidades adicionais, como definições, ajuda e contacto com o suporte.
\end{itemize}

Esta estrutura de navegação foi pensada para seguir padrões familiares aos utilizadores de aplicações móveis modernas, reduzindo o esforço cognitivo e aumentando a previsibilidade da interação.  
A \textit{Main Interface} representa, assim, o centro funcional da aplicação, equilibrando clareza informativa com acessibilidade e eficiência de uso.


\subsection{Interface de Perfil }


\begin{figure}[H]
    \centering
    \includegraphics[width=0.3\textwidth]{assets/Imagens/Perfil.png}
    \caption{Perfil}
\end{figure}

A interface de perfil tem como função centralizar e gerir as informações pessoais do utilizador associadas à aplicação.  
Esta secção foi desenhada para garantir facilidade de edição , alinhando-se com boas práticas de design e privacidade.

No topo da interface encontra-se a fotografia do utilizador, elemento visual que confere identidade e personalização à aplicação.  

Abaixo da fotografia são apresentados os campos de informação pessoal, todos organizados de forma hierárquica e agrupados logicamente.  
Os principais dados incluídos são:
\begin{itemize}
    \item \textbf{Nome completo};
    \item \textbf{Data de nascimento};
    \item \textbf{Endereço de e-mail};
    \item \textbf{Número de telemóvel};
    \item \textbf{Número de Identificação Fiscal (NIF)};
    \item \textbf{Morada completa};
    \item \textbf{Código postal};
\end{itemize}

Todos estes campos são editáveis, permitindo ao utilizador atualizar a sua informação de forma simples e rápida.  
Após qualquer modificação, a interface apresenta um aviso de confirmação e a possibilidade de desfazer alterações, assegurando o controlo total por parte do utilizador.

Além da edição de dados, esta secção inclui ainda opções para:
\begin{itemize}
    \item Alterar a fotografia do perfil;
    \item Gerir credenciais de login (por exemplo, redefinir palavra-passe);
    \item Configurar preferências de notificações e métodos de pagamento associados;
    \item Consultar políticas de privacidade e termos de utilização.
\end{itemize}

\subsection{Interface Adicionar/Comprar (\textit{Add or Purchase Interface})}




\begin{figure}[H]
    \centering
    \includegraphics[width=0.3\textwidth]{assets/Imagens/AdicionarComprar.png}
    \caption{Adicionar/Comprar}
\end{figure}

A interface \textit{Adicionar/Comprar} é acedida a partir do botão “Adicionar/Comprar” presente na interface principal.  
Esta secção da aplicação foi concebida para proporcionar ao utilizador um ponto de entrada rápido e intuitivo para duas ações distintas mas complementares: adicionar um lmente as opções disponíveis e o resultado esperincipais, dispostos de forma equilibrada e com rotulagem clara:
\begin{itemize}
    \item \textbf{Iniciar Leitura} — opção destinada a utilizadores que já possuem um cartão físico e pretendem associá-lo à aplicação.  
    Ao selecionar esta opção, o utilizador é direcionado para uma nova interface que ativa o sistema de leitura do cartão (via NFC ou câmara), permitindo capturar automaticamente os dados necessários à importação do passe para o dispositivo móvel.
    
    \item \textbf{Comprar} — opção que conduz o utilizador para a interface de compra digital, onde poderá escolher o tipo de bilhete ou passe a adquirir, efetuar o pagamento e armazenar o título digital diretamente na aplicação.  
    Esta funcionalidade será detalhada em secções posteriores do relatório.
\end{itemize}


Para além das duas opções principais, a interface apresenta uma breve descrição informativa que orienta o utilizador sobre a finalidade de cada botão, evitando ambiguidades e erros de navegação.  

A \textit{Interface Adicionar/Comprar} representa, assim, uma ponte entre o mundo físico e o digital, permitindo tanto a integração de passes já existentes como a aquisição imediata de novos títulos de transporte.  
O seu design minimalista e centrado na ação reforça a eficiência de uso e contribui para uma experiência de navegação coerente e sem fricção dentro da aplicação.


\subsection{Interface de Opções de Compra}

\begin{figure}[H]
    \centering
    \begin{subfigure}[t]{0.3\textwidth}
        \centering
        \includegraphics[width=\textwidth]{assets/Imagens/TiposBilhetes.png}
        \caption{Tipos de bilhetes}
    \end{subfigure}
    \begin{subfigure}[t]{0.3\textwidth}
        \centering
        \includegraphics[width=\textwidth]{assets/Imagens/CompraBilheteUnico.png}
        \caption{Compra de Bilhete Único}
    \end{subfigure}
    \caption{Interfaces de Compras}
\end{figure}


A \textit{Interface de Opções de Compra} representa o ponto central do processo de aquisição de títulos de transporte dentro da aplicação.
Esta interface é acedida após o utilizador selecionar a opção “Comprar” na \textit{Interface Adicionar/Comprar}.

O ecrã apresenta quatro botões de seleção principais, cada um correspondente a um tipo distinto de título de transporte:
\begin{itemize}
    \item \textbf{Bilhete Único} – destinado à compra de viagens pontuais, permitindo selecionar o número de deslocações desejadas.
    \item \textbf{Bilhete Diário} – permite adquirir bilhetes válidos por 24 horas, adequados a utilizadores ocasionais ou turistas.
    \item \textbf{Passe} – opção voltada para utilizadores frequentes, possibilitando a compra ou renovação de passes mensais (municipais ou metropolitanos).
    \item \textbf{Zapping} – modalidade que adiciona crédito em saldo pré-pago, permitindo ao utilizador pagar viagens de forma flexível, sem necessidade de bilhetes individuais.
\end{itemize}

Cada uma destas opções conduz o utilizador a uma interface específica, adaptada ao tipo de título escolhido.  
Por exemplo, ao selecionar “Bilhete Único”, o utilizador é levado para um ecrã onde pode indicar a quantidade de viagens desejadas; ao selecionar “Passe”, surgem opções para escolher o tipo de passe e o período de validade.

Na parte inferior da interface, é apresentada uma área de resumo da compra, que mostra em tempo real o valor total calculado de acordo com as opções selecionadas.  
Este componente fornece feedback imediato ao utilizador, permitindo-lhe ajustar facilmente a quantidade ou o tipo de bilhete antes de prosseguir para o pagamento.

Abaixo do resumo, o utilizador encontra a secção de métodos de pagamento.  
Aqui é possível:
\begin{itemize}
    \item Selecionar um método de pagamento disponível (MB Way, cartão multibanco, etc);
    \item Utilizar o método de pagamento predefinido, caso já exista um cartão bancário associado à conta;
    \item Alterar ou adicionar novos métodos de pagamento diretamente a partir desta interface.
\end{itemize}

A disposição visual privilegia a hierarquia da informação e a simplicidade de navegação, assegurando que o processo de compra possa ser concluído em poucos passos.  



\subsection{Interface de Informação do Cartão}


\begin{figure}[H]
    \centering
    \includegraphics[width=0.3\textwidth]{assets/Imagens/InfosPasse.png}
    \caption{Adicionar/Comprar}
\end{figure}


A interface é acedida ao selecionar um dos cartões ou passes exibidos na interface principal.  
O seu objetivo é apresentar de forma detalhada toda a informação relevante associada a esse título de transporte, bem como oferecer ações diretas de gestão e controlo do mesmo.  
Esta secção representa um dos núcleos funcionais mais importantes da aplicação, uma vez que combina informação, segurança e interação.

No topo do ecrã, é apresentada a identificação visual do cartão, incluindo o nome do utilizador e o tipo de título (por exemplo, \textit{Passe Navegante Municipal}, \textit{Bilhete Zapping} ou \textit{Bilhete Diário}).  
Logo abaixo, são exibidos os principais dados relacionados com o cartão:
\begin{itemize}
    \item Validade do título e data de expiração;
    \item Saldo ou número de viagens disponíveis;
    \item Histórico recente de utilizações;
    \item Estado atual do passe (ativo, expirado ou suspenso).
\end{itemize}

A interface oferece ainda várias ações diretas, que permitem ao utilizador gerir o cartão de forma autónoma:
\begin{itemize}
    \item \textbf{Cancelar Cartão} – Possibilita o cancelamento imediato do passe em caso de perda ou furto.  
    A ação é acompanhada por uma janela de confirmação, prevenindo erros acidentais e assegurando que o utilizador compreende as consequências da operação.
    \item \textbf{Recarregamento Automático} – permite ativar ou configurar recarregamentos automáticos de saldo ou passes mensais.  
    O utilizador pode definir um valor mínimo de saldo que desencadeia o carregamento, promovendo conveniência e evitando situações de bloqueio por falta de crédito.
    \item \textbf{Editar Informação} – opção destinada à atualização de dados associados ao cartão, como apelido, preferências de notificação ou método de pagamento.
\end{itemize}

Um dos elementos mais inovadores desta interface é a funcionalidade que permite ao utilizador utilizar o telemóvel como cartão físico.  
O utilizador pode aproximar o dispositivo das máquinas de validação nos transportes públicos, realizando o \textit{scan} de forma rápida e segura.  
Para garantir a usabilidade, o ecrã apresenta instruções claras e feedback visual (como uma animação ou sinal sonoro) que confirmam a validação bem-sucedida.


\section{Carteira Movel}
\begin{figure}[H]
    \centering
    \includegraphics[width=0.30\linewidth]{assets/Imagens/Wallet.png}
    \caption{Carteira Movel}
\end{figure}
Esta interface permite validar o passe utilizando somente o telefone, adicionando o cartão (tal como fazemos com os cartões de débito). Integrando o passe assim,
evita o esquecimento frequente do passe ou ate mesmo o furto.

\section{Widget no ecra principal}
\begin{figure}[H]
    \centering
    \includegraphics[width=0.30\linewidth]{assets/Imagens/EcraBloqueio.png}
    \caption{Ecra Bloqueio}
\end{figure}
Com este widget na tela de bloqueio, e possível ver o saldo atual do passe sem ter que abrir a aplicação.

\section{Notificacao}
\begin{figure}[H]
    \centering
    \includegraphics[width=0.30\linewidth]{assets/Imagens/Notificacao.png}
    \caption{Notificacao}
\end{figure}
A notificação é enviada consoante o valor que o utilizador escolher.