O objetivo da avaliação foi analisar a usibilidade do protótipo de baixa fidelidade desenvolvido, identificando dificuldades, erros e oportunidades de melhoria na interação dos utilizadores com a aplicação. Para isso, foi dado o protótipo a utlizadores que aplicaram o método "pensar alto", que consiste a expressar todoas os pensamentos sobre a utilização da aplicação, de modo a compreender o raciocínio e as emoções dos participantes durante a execução das tarefas.

\begin{figure}[H]
    \centering
    \includegraphics[width=0.3\textwidth]{assets/Imagens/Sessao_1.jpeg}
    \caption{Teste do prótótipo de baixa fidelidade com utilizador}
\end{figure}

Foram realalizados testes de usabilidade com 4 utilizadores, entre os 20 e 45 anos, todos utilizadores frequentes do passe Navegante. Cada participante recebeu uma breve explicação sobre o objetivo do teste e foi convidado a executar a tarefa principal de adquirir e validar o passe mensal através do protótipo. Durante a execução foi pedido aos participantes que verbalizassem a sua experiência, com cada sessão a ter uma duração média de 2 minutos. 



\begin{table}[H]
\centering
\caption{Resumo das observações dos testes de usabilidade}
\begin{tabularx}{\textwidth}{l X X X}
\toprule
\textbf{Utilizador} & \textbf{Tarefa realizada com sucesso} & \textbf{Escala de satisfação} & \textbf{Tempo de sessão} \\
\midrule
1º Utilizador & Sim & 4/5 & 1 min, 30 seg \\
\addlinespace
2º Utilizador & Sim & 4/5 & 2 min \\
\addlinespace
3º Utilizador & Sim & 3/5 & 1 min, 50 seg \\
\addlinespace
4º Utilizador & Sim & 4/5 & 2 min, 40 seg \\
\bottomrule
\end{tabularx}
\end{table}

Todos os utilizadores conseguiram desepenhar a tarefa com sucesso e tiveram as seguintes observações:

\begin{itemize}
    \item Os utilizadores compreenderam facilmente a navegação principal.
    \item Conseguiram identificar facilmente o botão de ajuda e o botão de perfil.
    \item Aprovaram a integração com a carteira móvel.
\end{itemize}

No entanto tiveram alguns problemas:

\begin{itemize}
    \item Confudiram as opções "Adicionar/Comprar".
    \item Acharam "confuso" os passos a seguir para adquirir o passe.
    \item Não obtiveram feeback visual de validação.
\end{itemize}

Ao pedir-mos sugestões de melhoria foi relatado o seguinte:

\begin{itemize}
    \item Recomendaram mostrar o método de pagamenteo concluido.
    \item Melhorar a maneira de adquirir o passe para ser mais direto.
\end{itemize}

De forma geral, os utilizadores compreenderam a estrutura da aplicação e conseguiram concluir a tarefa proposta. No entanto, surgiram dificuldades na distinção entre as ações “Adicionar” e “Comprar”, bem como na perceção do feedback após a validação do passe. A maioria valorizou o design simples e a integração com a carteira móvel.