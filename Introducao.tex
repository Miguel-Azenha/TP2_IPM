Na sequência do primeiro trabalho prático de Interação Pessoa-Máquina, este segundo relatório dá continuidade ao estudo iniciado sobre o uso do passe \textit{Navegante} e as dificuldades associadas à sua natureza física. No Trabalho Prático nº1 foram identificadas várias frustrações relacionadas com o uso do cartão físico, nomeadamente o esquecimento, a necessidade de procurar o cartão na mala ou carteira, e as filas de espera para efetuar o carregamento nas máquinas automáticas. Estes resultados evidenciaram uma oportunidade clara de melhoria através da digitalização do passe e da integração com sistemas móveis já utilizados no quotidiano dos cidadãos.

O presente trabalho cujo objetivo é transformar os resultados da fase de investigação em propostas concretas de interface e interação, que respondam às necessidades reais dos utilizadores identificadas anteriormente.

Nesta fase, são exploradas diferentes ideias e conceitos de design que procuram tornar a experiência de mobilidade mais fluida, prática e integrada. O foco é criar uma solução que elimine a dependência do cartão físico e que permita validar e gerir o passe diretamente através do telemóvel, aumentando a conveniência e reduzindo a fricção no uso diário.

O processo de ideação incluiu a elaboração de esboços e de um \textit{storyboard} que representam as principais interações do utilizador com a aplicação proposta. Com base nestes elementos, foi desenvolvido um protótipo de baixa fidelidade, que simula as principais funcionalidades da interface e permite visualizar o fluxo de utilização.

