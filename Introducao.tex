
A presente avaliação experimental e formal é realizada no âmbito da Unidade Curricular de Interação Pessoa--Máquina e constitui o 3º Trabalho Prático. Este trabalho dá continuidade às fases de investigação e design centradas no utilizador executadas nos trabalhos práticos anteriores.

\subsection{Definição do Problema}

O foco central deste trabalho insere-se na análise e superação da fricção induzida pela dependência do cartão físico do passe Navegante no quotidiano dos utilizadores de transportes públicos. A fase inicial de investigação identificou que a gestão do título de transporte físico cria múltiplos pontos de fricção no fluxo diário. Estes constrangimentos incluem:

\begin{enumerate}
    \item \textbf{Esquecimento}: A dependência do cartão físico é uma fonte significativa de frustração, levando a situações recorrentes de esquecimento do passe em casa.
    
    \item \textbf{Ineficiência na Validação}: A necessidade de procurar o cartão na mala ou no bolso resulta em tempos médios de procura observados entre 12 a 35 segundos, ou em alguns casos superiores caso o cartão esteja dentro de uma mochila ou mala. Esta fricção causa filas, stress e pequenos atrasos acumulados na entrada dos transportes, especialmente nos autocarros. Adicionalmente, os validadores nos autocarros falham frequentemente a leitura à primeira tentativa, exigindo que o utilizador volte a tentar.
    
    \item \textbf{Complexidade no Carregamento}: O processo de renovação do passe, efetuado maioritariamente em máquinas automáticas, Multibanco ou espaços Navegante, gera filas de espera e frustração, especialmente no início do mês ou em casos de avaria das máquinas.
    
    \item \textbf{Falta de Integração Digital}: Existe frustração por parte dos utilizadores devido à ausência de uma versão digital completa do passe e à falta de integração com sistemas de carteira móvel amplamente utilizados, como a \textit{Apple Wallet} ou \textit{Google Wallet}.
\end{enumerate}

Estes fatores de frustração e ineficiência representam um desafio significativo no fluxo quotidiano, com os participantes a valorizarem soluções digitais que eliminem a dependência do cartão físico e que permitam a validação direta através do telemóvel. A solução proposta  consiste num protótipo de alta fidelidade que visa digitalizar o passe e integrar funcionalidades essenciais, como carregamento automático e notificações de saldo baixo.

\subsection{Objetivos da Avaliação}

O presente estudo visa realizar uma avaliação rigorosa do protótipo desenvolvido. A metodologia adotada segue o formato de uma avaliação experimental e formal, comparando o protótipo com um sistema semelhante que atua como \textit{baseline}.

Os objetivos específicos desta avaliação são:

\begin{enumerate}
    \item \textbf{Avaliar a Usabilidade e a Experiência do Utilizador}: Mensurar o nível de eficácia, eficiência e satisfação, recorrendo a métricas ordinais e contínuas, durante a execução de tarefas-chave no protótipo de alta fidelidade.
    
    \item \textbf{Comparação com o Sistema \textit{Baseline}}: Determinar se o protótipo digital oferece uma melhoria estatisticamente significativa no desempenho do utilizador em comparação com a solução de referência existente ou com um sistema semelhante que realize a mesma tarefa.
    
    \item \textbf{Validação dos Requisitos de Design}: Verificar se as funcionalidades implementadas, como a integração digital do passe ou o fluxo de carregamento, satisfazem os requisitos identificados durante a fase de investigação.
\end{enumerate}

\subsection{Sistemas em Comparação e Metodologia}

O sistema avaliado consiste num protótipo de alta fidelidade de uma aplicação móvel para gestão e validação do passe Navegante, desenvolvido em Figma. Este sistema será comparado com um sistema de controlo que permita realizar a mesma tarefa principal de gestão e validação de títulos de transporte.

A avaliação será conduzida através de uma metodologia experimental rigorosa, incluindo a definição de variáveis dependentes (métricas ordinais e contínuas), o recrutamento de 10 a 16 participantes e a análise estatística dos resultados, conforme detalhado no capítulo seguinte.
