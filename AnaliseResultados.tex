\subsection{Análise de Descritiva}
As perguntas feitas seguiram o seguinte formato: \cite{martins2015european}
\subsubsection{Satisfação SUS}
\paragraph{Fórmula para o SUS}
Para cada um dos 10 participantes, calcula-se uma pontuação individual:
\begin{enumerate}
    \item \textbf{Perguntas Ímpares}: Subtrair 1 à resposta do utilizador. ($R-1$)
    \item \textbf{Perguntas Pares}: Subtrair à resposta do utilizador 5. ($5-R$)
    \item \textbf{Soma Total}: Somar os resultados dos passos 1 e 2.
    \item \textbf{Multiplicador}: Multiplicar a soma por 2.5 para obter a escala de 0-100.
\end{enumerate}

Os participantes responderam as mesmas perguntas tanto para a aplicação atual como para 
o protótipo demonstrado.

\subsubsection*{Resultados App Atual}
\paragraph{Participante nº1}
\textit{Respostas: 3, 2, 2, 1, 3, 3, 5, 1, 4, 1} 

\textbf{Perguntas Ímpares}
\begin{align*}
    3-1 &= 2 \\
    2-1 &= 1 \\
    3-1 &= 2 \\
    5-1 &= 4 \\
    4-1 &= 3 \\
\Sigma &= 2+1+2+4+3 = 12
\end{align*}

\textbf{Perguntas Pares}
\begin{align*}
    5-2 &= 3 \\
    5-1 &= 4 \\
    5-3 &= 2 \\
    5-1 &= 4 \\
    5-1 &= 4 \\
    \Sigma &= 3+4+2+4+4 = 17
\end{align*}

\textbf{Cálculo Final}
\begin{itemize}
    \item Soma Total $= 12 + 17 = 29$
    \item Pontução SUS $= 29 * 2.5 = 72.5$
\end{itemize}

Fazendo estes cálculos para todos os participantes, obtiveram-se os seguintes resultados:
\begin{table}[h]
    \centering
    % Sugestão: Verifica se queres manter a legenda igual à anterior ou mudar para algo como "Aplicação Antiga"
    \renewcommand{\arraystretch}{1.2} 
    \begin{tabular}{lcccc}
        \toprule
        \textbf{Part.} & \textbf{Ímpares} & \textbf{Pares} & \textbf{Soma} & \textbf{SUS} \\
        \midrule
        1  & 12 & 17 & 29 & 72,5 \\
        2  & 12 & 17 & 29 & 72,5 \\
        3  & 13 & 20 & 33 & 82,5 \\
        4  & 15 & 20 & 35 & 87,5 \\
        5  & 14 & 17 & 31 & 77,5 \\
        6  & 16 & 19 & 35 & 87,5 \\
        7  & 14 & 17 & 31 & 77,5 \\
        8  & 4  & 7  & 11 & 27,5 \\
        9  & 9  & 11 & 20 & 50   \\
        10 & 10 & 9  & 19 & 47,5 \\
        \bottomrule
    \end{tabular}
    \caption{Tabela de Pontuação dos Participantes para a Aplicação Atual} 
    \label{tab:pontuacao-app} % Mudei o label para não dar erro de duplicado
\end{table}


\[
\text{Média: }
M = \frac{\Sigma x}{N} = \frac{682,5}{10} = 68,5
\]

\subsubsection*{Resultados Protótipo}
Seguindo os mesmos cáclulos feitos acima:

\begin{table}[h]
    \centering
    \renewcommand{\arraystretch}{1.2} 
    \begin{tabular}{lcccc}
        \toprule
        \textbf{Part.} & \textbf{Ímpares} & \textbf{Pares} & \textbf{Soma} & \textbf{SUS} \\
        \midrule
        1  & 18 & 18 & 36 & 90   \\
        2  & 20 & 18 & 38 & 95   \\
        3  & 18 & 19 & 37 & 92,5 \\
        4  & 19 & 20 & 39 & 97,5 \\
        5  & 18 & 20 & 38 & 95   \\
        6  & 20 & 20 & 40 & 100  \\
        7  & 16 & 18 & 34 & 85   \\
        8  & 14 & 16 & 30 & 75   \\
        9  & 17 & 16 & 33 & 82,5 \\
        10 & 16 & 15 & 31 & 77,5 \\
        \bottomrule
    \end{tabular}
    \caption{Tabela de Pontuação dos Participantes para a Aplicação Atual}
    \label{tab:pontuacao-prot}
\end{table}


\[
\text{Média: }
M = \frac{\Sigma x}{N} = \frac{890}{10} = 89
\]

\subsubsection{Teste de Hipótese}
Para provar estastisticamente que a melhoria não foi "sorte", calcula-se o
\textbf{Paired T-Test}

\subsubsection*{Fórmula}
\[
t = \frac{\overline{d}}{s_d / \sqrt{n}}
\]
Onde:
\begin{itemize}
    \item $\overline{d}$ = Média das diferenças entre os dois sistemas
    \item $s_d$ = Desvio padrão das diferenças.
    \item $n$ = Número de Participantes (10).
\end{itemize}

Começa-se por calcular a diferença SUS para cada participante:
\begin{table}[h]
    \centering
    \label{tab:comparacao-sus}
    \renewcommand{\arraystretch}{1.2} 
    \begin{tabular}{lccc}
        \toprule
        \textbf{Part.} & \textbf{App Atual} & \textbf{Protótipo} & \textbf{Diferença} \\
        \midrule
        1  & 72,5 & 90   & +17,5 \\
        2  & 72,5 & 95   & +22,5 \\
        3  & 82,5 & 92,5 & +10,0 \\
        4  & 87,5 & 97,5 & +10,0 \\
        5  & 77,5 & 95   & +17,5 \\
        6  & 87,5 & 100  & +12,5 \\
        7  & 77,5 & 85   & +7,5  \\
        8  & 27,5 & 75   & +47,5 \\
        9  & 50   & 82,5 & +32,5 \\
        10 & 47,5 & 77,5 & +30,0 \\
        \midrule
        \textbf{Média} & \textbf{68,3} & \textbf{89,0} & \textbf{+20,75} \\
        \bottomrule
    \end{tabular}
    \caption{Comparação de pontuação SUS entre a Aplicação Atual e o Protótipo}
\end{table}

De seguida, calcula-se o Desvio Padrão ($s_d$):
\[
s_d = \sqrt{\frac{\Sigma (d_i - \overline{d})^2}{n-1}}
\]

\begin{align*}
    \text{P1:} (17,5 - 20,75)^2 &= (-3,35)^2 &= 10,56 \\
    \text{P2:} (22,5 - 20,75)^2 &= (1,75)^2 &= 3,06 \\ 
    \text{P3:} (10,0-20,75)^2 &= (-10,75)^2 &= 115,56 \\ 
    \text{P4:} (10,0-20,75)^2 &= (-10,75)^2 &= 115,56 \\ 
    \text{P5:} (17,5 - 20,75)^2 &= (-3,25)^2 &= 10,56 \\
    \text{P6:} (12,5 - 20,75)^2 &= (-8,25)^2 &= 68,06 \\
    \text{P7:} (7,5 - 20,75)^2 &= (-13,25)^2 &= 175,56 \\
    \text{P8:} (47,5 - 20,75)^2 &= (26,75)^2 &= 715,56 \\
    \text{P9:} (32,5 - 20,75)^2 &= (-11,75)^2 &= 138,06 \\
    \text{P10:} (30,0 - 20,75)^2 &= (9,25)^2 &= 85,56 \\ 
    \Sigma &= 1438,125
\end{align*}

\begin{align*}
    \text{Variância}(s^2) &= \frac{1438,125}{10-1} &\approx 159,79 \\ 
    s_d &= \sqrt{157,79} &\approx 12,64
\end{align*}

O desvio padrão de 12,64 indica que existe uma variabilidade considerável na forma como as
pessoas reagiram à mudança.

Depois, calcula-se o Erro Padrão ($SE$):
\[
SE = \frac{s_d}{\sqrt{n}} = \frac{12,64}{\sqrt{10}} = \frac{12,64}{3,162} \approx 4,00
\]

Calcula-se agora o Valor T (\textit{t-value})
\[
t = \frac{\overline{d}}{SE} = \frac{20,75}{4,00} \approx 5,19
\]

Por fim, calcula-se o valor P (\textit{p-value})

Para encontrar o \textit{p-value}, tem que se consultar a Tabela de Distribuição T com:
\begin{enumerate}
    \item \textbf{Graus de Liberdade} $(df): n-1 = 9$
    \item \textbf{Valor T}: 5,19
\end{enumerate}

Analisando na tabela T para $df = 9$:
\begin{itemize}
    \item Para $p = 0.05$, o $t$ crítico é \textbf{1,833}.
    \item Para $p = 0.001$, o $t$ crítico é \textbf{4,297}
\end{itemize}

Como o valor t é Superior a 4,297, o \textit{p-value} é menor do que 0.001.