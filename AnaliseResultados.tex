\subsection{Análise de Descritiva}
As perguntas feitas seguiram o seguinte formato: \cite{martins2015european}
\subsubsection{Satisfação SUS}
\paragraph{Fórmula para o SUS}
Para cada um dos 10 participantes, calcula-se uma pontuação individual:
\begin{enumerate}
    \item \textbf{Perguntas Ímpares}: Subtrair 1 à resposta do utilizador. ($R-1$)
    \item \textbf{Perguntas Pares}: Subtrair à resposta do utilizador 5. ($5-R$)
    \item \textbf{Soma Total}: Somar os resultados dos passos 1 e 2.
    \item \textbf{Multiplicador}: Multiplicar a soma por 2.5 para obter a escala de 0-100.
\end{enumerate}

Os participantes responderam as mesmas perguntas tanto para a aplicação atual como para 
o protótipo demonstrado.

\subsubsection*{Resultados App Atual}
\paragraph{Participante nº1}
\textit{Respostas: 3, 2, 2, 1, 3, 3, 5, 1, 4, 1} 

\textbf{Perguntas Ímpares}
\begin{align*}
    3-1 &= 2 \\
    2-1 &= 1 \\
    3-1 &= 2 \\
    5-1 &= 4 \\
    4-1 &= 3 \\
\Sigma &= 2+1+2+4+3 = 12
\end{align*}

\textbf{Perguntas Pares}
\begin{align*}
    5-2 &= 3 \\
    5-1 &= 4 \\
    5-3 &= 2 \\
    5-1 &= 4 \\
    5-1 &= 4 \\
    \Sigma &= 3+4+2+4+4 = 17
\end{align*}

\textbf{Cálculo Final}
\begin{itemize}
    \item Soma Total $= 12 + 17 = 29$
    \item Pontução SUS $= 29 * 2.5 = 72.5$
\end{itemize}

Fazendo estes cálculos para todos os participantes, obtiveram-se os seguintes resultados:
\begin{table}[h]
\centering
\begin{tabular}{|l|c|c|c|c|}
\hline
\textbf{Participante} & \textbf{Ímpares} & \textbf{Pares} & \textbf{Soma} & \textbf{SUS} \\
\hline
1 & 12 & 17 & 29 & 72,5 \\
\hline
2 & 12 & 17 & 29 & 72,5 \\
\hline
3 & 13 & 20 & 33 & 82,5 \\
\hline
4 & 15 & 20 & 35 & 87,5 \\
\hline
5 & 14 & 17 & 31 & 77,5 \\
\hline
6 & 16 & 19 & 35 & 87,5 \\
\hline
7 & 14 & 17 & 31 & 77,5 \\
\hline
8 & 4 & 7 & 11 & 27,5 \\
\hline
9 & 9 & 11 & 20 & 50 \\
\hline
10 & 10 & 9 & 19 & 47,5 \\
\hline
\end{tabular}
\caption{Tabela de Pontuação dos Participantes para a Aplicação Atual }
\label{tab:pontuacao-app}
\end{table}


\[
\text{Média: }
M = \frac{\Sigma x}{N} = \frac{682,5}{10} = 68,5
\]

\subsubsection*{Resultados Protótipo}
Seguindo os mesmos cáclulos feitos acima:

\begin{table}[h]
\centering
\begin{tabular}{|l|c|c|c|c|}
\hline
\textbf{Participante} & \textbf{Ímpares} & \textbf{Pares} & \textbf{Soma} & \textbf{SUS} \\
\hline
1 & 18 & 18 & 36 & 90 \\
\hline
2 & 20 & 18 & 38 & 95 \\
\hline
3 & 18 & 19 & 37 & 92,5 \\
\hline
4 & 19 & 20 & 39 & 97,5 \\
\hline
5 & 18 & 20 & 38 & 95 \\
\hline
6 & 20 & 20 & 40 & 100 \\
\hline
7 & 16 & 18 & 34 & 85 \\
\hline
8 & 14 & 16 & 30 & 75 \\
\hline
9 & 17 & 16 & 33 & 82,5 \\
\hline
10 & 16 & 15 & 31 & 77,5 \\
\hline
\end{tabular}
\caption{Tabela de Pontuação dos Participantes para a Aplicação Atual }
\label{tab:pontuacao-app}
\end{table}


\[
\text{Média: }
M = \frac{\Sigma x}{N} = \frac{890}{10} = 89
\]
